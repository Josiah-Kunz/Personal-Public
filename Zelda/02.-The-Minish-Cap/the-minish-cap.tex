\input{../preamble}

% ================

\newpage
\graphic{filename=Preliminary/box-art, scale=1}

% ================

\newpage

% Unboxing
\graphic{filename=Preliminary/unboxing, caption={Just got a package in the mail! Wonder what it is...!}}

\graphic{filename=Preliminary/hardware, caption={Hardware. \textbf{Left}: cartridge bought from zelda-shop.com for \$30. \textbf{Middle}: Gameboy Advance bought from Press~Start in Jacksonville, IL for \$50. No backlit screen since I really wanted it to be on original hardware. On the other hand, I never owned a GBA, so I guess I could've done whatever I wanted. \textbf{Right}: GBA case the guy at Press~Start threw in for \$2. I'm pretty sure it was supposed to be marketed towards girls, but hey, a case is a case. \textbf{Total}: \$82 (out of the \$100 I got from Dad for my 34th birthday).}}

\graphic{filename=Preliminary/worm-light, caption={Additional hardware. I really needed something to see the screen and this was the hardware of the times. I could have backlit the GBA for about \$80, but that certainly wasn't available in the mid 2000s. I don't remember how much it cost, but it was cheap.}}

% ================

\newpage

% Completion overview
\begin{figure}[H]
	\centering
	\includegraphics[scale=2]{Completion/overview}
	\caption{Overview of completion goals.}
\end{figure}
	
\newpage

% ToC
\etocsettocstyle{}{}
\localtableofcontents

% ================

\newpage
\sect{Life Gauge}

\graphic{filename=Completion/life}

% ================

\newpage
\sect{Homes}

\graphic{filename=Screenshots/farore-house, caption={
	The first house was for Farore. It was her Kinstone that opened the house up, so it only felt right to give it to her.
}}

% ================

\newpage
\sect{Review}
\begin{itemize}
	\item{First impression: I'm guessing this is a short game. Also, I'm not too excited to play it because the darkness of the screen. We'll see if that's a dealbreaker, though.}
	\item{It's neat that about half the dungeons are in your Minish form.}
	\item{Speaking of dungeons, I like that you use multiple items and strategies. It's not just ``we found weapon X in the dungeon so that's how you kill the boss'' (although that still happens sometimes).}
	\item{Some of the puzzles are just too hard. And I don't mean you need a big brain, I mean there's no guidance. Like, so little sense of direction that you get discouraged and want to give up.}
	\item{There's a softlock in Dark Hyrule Castle that really sucks.}
	\item{Vaati was actually tough for me to figure out. Maybe I just suck, but the three chimes timing was challenging enough (without losing hearts) and the boss fight has some neat mechanics.}
\end{itemize}

% ================

\newpage
\sect{Questions and Lore}
\begin{itemize}
	\item{Although they're from the sky, clearly the Minish (Picori) are not the Skyloftians. Perhaps, though, the people from the Fortress of Winds were. The boss of the Fortress of Winds was Mazaal, who definitely resembled the tech of the ancient Lanayru robots in Skyward Sword.
	\graphic{filename=Other/mazaal, caption={\textbf{Top left:} Mazaal from the Fortress of Winds (Minish Cap). \textbf{Bottom left and right:} ancient robot enemies Armos and Beamos from Lanayru Desert (Skyward Sword).}}
	Heck, Mazaal's hands even had the same gimmick as Armos (kill the hidden ball/power source thing).
	}
\end{itemize}

% ================


\input{../postamble}